%*********************************************************************
% RUC Thesis: 中国人民大学本科生毕业论文模板
% Copyright 2022  Qiu Renxiang
% 2022/03/20 v1.2 beta
%
%本文编写的依据是中国人民大学教务处2021年修订的《本科生毕业论文(设计)指导手册》(http://jiaowu.ruc.edu.cn/wjxz6/sjjx3/828d0d99e9bd4686b41b3526fe553b73.htm)以及本科毕业论文模板(http://jiaowu.ruc.edu.cn/wjxz6/sjjx3/d5aba7b8b1574080a34af94916c2464d.htm)
%本文在2020年的1.0版本(2017年版论文指导手册)基础上进行修改完善:
% 1. 首页格式依据最新指导手册进行更新
% 2. 修改了上一版本部分字号以及字体使用不当的问题(针对多级标题和目录进行大量优化)
% 3. 在参考文献引用上采用了GB/T7714-2015标准,按照著者-出版年制排序,如需更改请参考手册:https://ctan.math.illinois.edu/macros/latex/contrib/biblatex-contrib/biblatex-gb7714-2015/biblatex-gb7714-2015.pdf
% 4. 脚注采用圆形脚注
% 5. 添加了附录页和致谢页
% 6. 原创性声明和使用授权说明请自行打印并插入(1.2已更新)
% 7. 官方本科毕业论文Word模板有较多地方与指导手册冲突,如遇与本模板不同,一切以指导手册为准
% 8. (1.2)修改了关于参考文献的细节。


%如果你有任何建议和疑问请发送邮件至qrx_math@ruc.edu.cn,协助我做的更好。
% 重要提示:
%   1. 请确保使用 UTF-8 编码保存
%   2. 请使用 XeLaTeX编译
%   3. 修改、使用、发布本文档请务必遵循 LaTeX Project Public License
%   4. 不需要的注释可以尽情删除
%*********************************************************************
\documentclass[12pt,UTF8]{ctexart}
\usepackage{amsmath}
\usepackage{amssymb}
\usepackage{float}
\usepackage{graphicx}
\usepackage{epstopdf}
\usepackage{booktabs}
\usepackage{multirow}
\usepackage{geometry}
\usepackage{appendix}
\usepackage{hyperref}
\usepackage{enumerate}
\usepackage{threeparttable}
\usepackage{fancyhdr}
\usepackage{setspace}
\usepackage[T1]{fontenc}
\usepackage{mathptmx}%Times New Roman字体
\usepackage{titletoc}
\usepackage{fontspec}
\usepackage{pdfpages}
\usepackage{tikz}
\usepackage{etoolbox}
\usepackage{xcolor}
\usepackage{caption}
\usepackage{array}
\usepackage{enumitem}
\usepackage{titlesec}



\usepackage[backend=biber,style=gb7714-2015ay,gbalign =right,gbpub=true,mergedate=none]{biblatex}
%biblatex宏包的参考文献数据源加载方式
\addbibresource{references.bib}

\captionsetup{font={small},labelfont={bf}} 
\captionsetup[table]{skip=3pt}
\captionsetup[figure]{skip=3pt}
%圆形脚注
\usepackage{pifont}
\usepackage[perpage,symbol*]{footmisc}
\DefineFNsymbols{circled}{{\ding{192}}{\ding{193}}{\ding{194}}
{\ding{195}}{\ding{196}}{\ding{197}}{\ding{198}}{\ding{199}}{\ding{200}}{\ding{201}}}
\setfnsymbol{circled}

%图标编号按章节排序(可选)
% \renewcommand {\thetable} {\thesection{}.\arabic{table}}
% \renewcommand {\thefigure} {\thesection{}.\arabic{figure}}

%修改目录

\titlecontents{section} % set formatting for \section -
                        % \subsection must be formatted separately 
[2.3em]                 % adjust left margin
{\heiti}             % font formatting
{\contentslabel{2.3em}} % section label and offset
{\hspace*{-2.3em}}
{\titlerule*[1pc]{.}\contentspage}

%以及小标题样式
% 标题:各层标题均单独占行。
% 一级标题:三号黑体(英文Times New Roman),加粗,居中,单倍行距,段前1行,段后1行,序号后空2个汉字符接标题内容,末尾不加标点,写法为“1 …”;
% 二级标题:四号黑体(英文Times New Roman),加粗,顶左,单倍行距,段前1行,段后1行,序号后空2个汉字符接标题内容,末尾不加标点,写法为“1.1 …”;
% 三级标题:小四号黑体(英文Times New Roman),加粗顶左,单倍行距,段前1行,段后1行,序号后空2个汉字符接标题内容,末尾不加标点,写法为“1.1.1  …”;
% 四级标题:五号黑体(英文Times New Roman),加粗顶左,单倍行距,段前1行,段后1行,序号后空2个汉字符接标题内容,末尾不加标点,写法为“1.1.1.1  …”。
\titleformat{\section}{\centering \zihao{3} \heiti \bfseries}{\thesection}{1em}{}

\titleformat{\subsection}{\flushleft \zihao{4} \heiti \bfseries}{\thesubsection}{1em}{}
\titleformat{\subsubsection}{\flushleft \zihao{-4} \heiti \bfseries}{\thesubsubsection}{1em}{}

%本科学生毕业论文要求纵向打印,页边距的要求为:
%上(T):2 cm
%下(B):2 cm
% 左(L):1.5 cm
% 右(R):1.5 cm
% 装订线(T):0.5 cm
% 装订线位置(T):左
\geometry{a4paper,left = 2cm,right = 1.5cm,top = 2cm,bottom= 2cm}


% 页眉:学校标志(教务处主页提供下载):
% 高度为0.98 cm,宽度为4.13 cm,居中放置。
% 页码采用10.5号宋体字,居中放置,格式为:第1页。
\pagestyle{fancy}
\lhead{}
\chead{\includegraphics[width = 4.13cm,height = 0.98cm]{head.png}}
\rhead{}
\lfoot{}
\rfoot{}

%目录和引用超链接
\hypersetup{
colorlinks=true,
linkcolor=black,
citecolor=black
}








\begin{document}

\newcommand\dunderline[3][-1pt]{{%
  \setbox0=\hbox{#3}
  \ooalign{\copy0\cr\rule[\dimexpr#1-#2\relax]{\wd0}{#2}}}}

% 2.3.1首页
% 2.3.1.1论文编码: 12号黑体字,加粗,置顶,居右。
% 2.3.1.2文头:“中国人民大学本科毕业论文(设计)”,在论文编码下一行, 28号黑体字,加粗,居中。
% 2.3.1.3论文题名:论文文头下,隔一行,28号黑体字,加粗,居中。
% 2.3.1.4论文副题名:居中排印在论文题名下,20号黑体字,加粗,副题名前加特殊符号中“长划线”。
% 2.3.1.5作者、学院、专业、年级、学号、指导教师、论文成绩、日期:论文标题下隔六行,依次排印在论文副题名下,各占一行,距左端空5格,名称后用“:”, 20号黑体字,加粗,内容下需要加下划线,内容置于下划线中部,两端对齐。

\begin{titlepage}
\thispagestyle{fancy}
\cfoot{}
  \begin{flushright}
    \heiti \textbf{论文编码:RUC-BK-050101-2018000000}
  \end{flushright}



  \fontsize{28pt}{\baselineskip}\textbf{\heiti{中国人民大学本科毕业论文(设计)}}

  \vspace{24mm}
  \centering
  \fontsize{28pt}{\baselineskip}\textbf{\heiti 对xxx的研究}

  \vspace{3mm}

  \begin{spacing}{1.2}
    \LARGE\selectfont{\textbf{\heiti ——以xx为例}}
  \end{spacing}

  \vspace{64mm}

  \flushleft
  \begin{spacing}{1.3}
    \hspace{27mm}\heiti\LARGE\selectfont{\textbf{作\hspace{14mm}者:}\dunderline[-10pt]{1pt}{\makebox[78mm][c]{邱任翔}}}
  
    \hspace{27mm}\heiti\LARGE\selectfont{\textbf{学\hspace{14mm}院:}\dunderline[-10pt]{1pt}{\makebox[78mm][c]{信息学院}}}

    \hspace{27mm}\heiti\LARGE\selectfont{\textbf{专\hspace{14mm}业:}\dunderline[-10pt]{1pt}{\makebox[78mm][c]{财税数学实验班}}}

    \hspace{27mm}\heiti\LARGE\selectfont{\textbf{年\hspace{14mm}级:}\dunderline[-10pt]{1pt}{\makebox[78mm][c]{2018级}}}

    \hspace{27mm}\heiti\LARGE\selectfont{\textbf{学\hspace{14mm}号:}\dunderline[-10pt]{1pt}{\makebox[78mm][c]{2018000000}}}
    

    \hspace{27mm}\heiti\LARGE\selectfont{\textbf{指导教师:}\dunderline[-10pt]{1pt}{\makebox[78mm][c]{XXX}}}
    
    \hspace{27mm}\heiti\LARGE\selectfont{\textbf{论文成绩:}\dunderline[-10pt]{1pt}{\makebox[78mm][c]{}}}
    
    \hspace{27mm}\heiti\LARGE\selectfont{\textbf{完成日期:}\dunderline[-10pt]{1pt}{\makebox[78mm][c]{2022.03.02}}}
  \end{spacing}

  \vspace{25mm}

\end{titlepage}

\newpage
\thispagestyle{fancy}
\cfoot{}

\mbox{}
\vspace{14mm}

\begin{center}
    \heiti \zihao{3} {中国人民大学学位论文原创性声明和使用授权说明}

\textbf{\songti \zihao{4} 原创性声明}
\end{center}
\vspace{5mm}

本人郑重声明:所呈交的学位论文,是本人在导师的指导下,独立进行研究工作所取得的成果。除文中已经注明引用的内容外,本论文不含任何其他个人或集体已经发表或撰写过的作品或成果。对本文的研究做出重要贡献的个人和集体,均已在文中以明确方式标明。

\vspace{10mm}
\begin{spacing}{1.8}
\begin{flushright}
论文作者签名:\hspace{21mm}       

日期:\qquad 年\qquad 月\qquad 日
\end{flushright}
\end{spacing}

\vspace{34mm}

\begin{center}
\textbf{\songti \zihao{4} 学位论文使用授权说明}
\end{center}
\vspace{5mm}

本人完全了解中国人民大学关于收集、保存、使用学位论文的规定,即:

\begin{description}
\item[\huge \labelitemi] 按照学校要求提交学位论文的印刷本和电子版本;
\item[\huge \labelitemi] 学校可以公布论文的全部或部分内容,可以采用影印、缩印或其他复制手段保存论文。
\end{description}

\vspace{10mm}

\begin{spacing}{1.8}
\begin{flushright}
论文作者签名:\hspace{21mm} 

指导教师签名:\hspace{21mm} 

日期:\qquad 年\qquad 月\qquad 日
\end{flushright}
\end{spacing}


% 中文摘要及关键词:
% “摘要”,单倍行距,段前1行,段后1行,16号黑体字,加粗,居中。
% 摘要内容, 12号宋体字,起行空两格,回行顶格, 1.5倍行距。
% “关键词”,摘要下隔一行,左端顶格,12号黑体字,加粗,后面加“:”。
% 关键词内容,直接放在“:”之后,词间间隔3格,12号宋体字。

\newpage
\renewcommand{\abstractname}{\textbf{\Large \heiti 摘要}}
\renewenvironment{abstract}{%
    \par\small
    \noindent\mbox{}\hfill{\bfseries \abstractname}\hfill\mbox{}\par
    \vskip 2.5ex}{\par\vskip 2.5ex} 
%中文摘要及关键词放在扉页一、外文摘要及关键词放在扉二,页码编排为Ⅰ,Ⅱ,设置页眉
\begin{abstract}
%1.5倍行距
\begin{onehalfspace}
这里是中文摘要。在对论文进行总结的基础上,用简单、明确、易懂、精辟的语言对全文内容加以概括,提取论文的主要信息。
\\[12pt]
\textbf{\textbf{\heiti 关键词:}}关键词1 \quad 关键词2 \quad 关键词3
\end{onehalfspace}
\end{abstract}
\setcounter{page}{1}
\pagenumbering{Roman}
\cfoot{\footnotesize \thepage}
\newpage


% 外文摘要及关键词:
% Abstract,单倍行距,段前1行,段后1行,16号Times New Roman字,加粗,居中。
% 内容使用12号Times New Roman字,起行空两格,回行顶格,两倍行距。
% Key Words,英文摘要内容下隔一行,左端顶格,12号Times New Roman字,
% 加粗,后面加“:”。
% 关键词内容,直接放在“:”之后,互相之间间隔3格,12号Times New Roman字。

\newcommand{\enabstractname}{\textbf{\Large Abstract}}
\newenvironment{enabstract}{%
    \par\small
    \noindent\mbox{}\hfill{\bfseries \enabstractname}\hfill\mbox{}\par
    \vskip 2.5ex}{\par\vskip 2.5ex}  
\begin{enabstract}
\begin{doublespace}
This is abstract. Use simple, clear, understandable, incisive language to summarize the full text content, extract the main information of the paper.
\\[12pt]
\textbf{Keywords:}key word 1 \quad  key word 2  \quad key word3
\end{doublespace}
\end{enabstract}
\cfoot{\thepage}
\newpage
\cfoot{}
\renewcommand\contentsname{内容目录}
\renewcommand\listfigurename{插\ 图}
\renewcommand\listtablename{表\ 格}
\tableofcontents
%不需要图表目录可以删去下面两行
\listoffigures
\listoftables
\newpage
\setcounter{page}{1}
\pagenumbering{arabic}
\pagestyle{fancy}
\cfoot{\zihao{5} 第 \thepage 页}
% 按照自然段依次排列,每段起行空两格,回行顶格。12号宋体字,(重点文句,12号宋体字,加粗),1.25倍行距。
%正文、注释双面打印,编排页码,自第1页起,设置页眉。
\begin{spacing}{1.25}
\section{绪论}
这里是一段正文示例\footnote{这是脚注。}。

这是分项编号:
\begin{enumerate}[label = (\arabic*)]
    \item 111
    \item 222
        \begin{enumerate}[label=\textcircled{\arabic*}]
            \item 1111
            \item 2222
        \end{enumerate}
    \item 333
\end{enumerate}

这是一张带备注的三线表:
\begin{table}[H]
\centering
\caption{2013年-2019年人均交通支出和人均可支配收入}
\label{tab:trans}
\begin{threeparttable}
\begin{tabular}{@{}cccc@{}}
\toprule
\textbf{年份} & \textbf{人均交通消费支出(元)} & \textbf{人均可支配收入(元)} & \textbf{比值} \\ \midrule
2013        & 1627                 & 18311               & 0.089       \\
2014        & 1869                 & 20167               & 0.093       \\
2015        & 2087                 & 21966               & 0.095       \\
2016        & 2338                 & 23821               & 0.098       \\
2017        & 2499                 & 25974               & 0.096       \\
2018        & 2675                 & 28228               & 0.095       \\
2019        & 2862                 & 30733               & 0.093       \\ \bottomrule
\end{tabular}
\begin{tablenotes}
    \item \footnotesize 数据来源:国家统计局
\end{tablenotes}
\end{threeparttable}
\end{table}

这是一张带备注的图:
\begin{figure}[H]
    \centering
    \includegraphics[scale = 0.5]{title.jpg}
    \caption{中国人民大学校徽}
    \label{fig:age}
    {\fontsize{8pt}{10pt}\selectfont\parbox{0.7\textwidth}{来源:中国人民大学教务处下载中心}}
\end{figure}
\subsection{研究意义}
\subsubsection{理论意义}

\section{文献综述}

\vspace*{\fill}
\begin{flushright}
    \heiti \large 作者签名:\dunderline[-10pt]{1pt}{\makebox[38mm][c]{}}
  \end{flushright}
\newpage
\end{spacing}
\nocite{*}
\addcontentsline{toc}{section}{\heiti 参考文献}
\printbibliography[heading=bibliography,title=参考文献]
\newpage
\addcontentsline{toc}{section}{\heiti 附录}
\section*{附录}
\appendix
\newpage
\addcontentsline{toc}{section}{\heiti 致谢}
\section*{致谢}
\end{document}
